
Les fonctions utilitaires servent à simplifier et centraliser des opérations courantes dans l'application. Voici leur rôle :
\begin{itemize}
    \item \texttt{get\_header\_style()} : Retourne un dictionnaire de styles CSS pour les en-têtes avec un dégradé linéaire et une mise en forme moderne.
    \item \texttt{fmt\_date(ts)} : Prend une date (timestamp) et la formate sous la forme "jour-mois-année" pour l'affichage.
    \item \texttt{make\_badges(start, end)} : Génère deux badges Dash affichant les dates de début et de fin d'une plage sélectionnée.
    \item \texttt{normalize\_range(start, end)} : Prend deux dates et les ordonne chronologiquement, en gérant les cas où l'utilisateur inverserait début et fin.
    \item \texttt{make\_stocks\_fig(df)} : Crée un graphique linéaire Plotly à partir d'un DataFrame de cours boursiers, en utilisant les paramètres de style définis.
\end{itemize}

Les composants réutilisables sont des fonctions qui construisent des blocs d'interface graphique ("cards", sélecteurs, etc.) pour éviter la répétition de code et garantir une cohérence visuelle :
\begin{itemize}
    \item \texttt{gradient\_card\_header(content)} : Génère un en-tête de carte avec le dégradé linéaire et le style centralisé.
    \item \texttt{card(gid, title, height)} : Crée une carte contenant un graphique (par exemple, pour les séries temporelles) avec un titre et un badge, le tout stylisé.
    \item \texttt{custom\_card(content, title, body\_style, card\_class)} : Carte générique permettant d'insérer n'importe quel contenu, avec ou sans titre, et de personnaliser le style du corps.
    \item \texttt{date\_card()} : Carte contenant le sélecteur de plage de dates et les badges min/max, utilisée dans le premier onglet.
    \item \texttt{metric\_select()} : Génère le sélecteur déroulant pour choisir la métrique (population, PIB, espérance de vie) dans le second onglet, avec un label stylisé.
    \item \texttt{gapminder\_card()} : Carte contenant le graphique des indicateurs mondiaux, avec un fond en dégradé.
\end{itemize}
\section*{Explication du code Dash -- Exercice 8.2}

Ce code met en place une application web interactive avec Dash et Plotly, permettant de visualiser des données financières et des indicateurs mondiaux à travers une interface moderne et réactive.

\subsection*{Structure générale}
Le code est organisé en plusieurs parties :
\begin{itemize}
    \item \textbf{Constantes et styles} : Définition des couleurs, styles graphiques, labels, et options utilisées dans toute l'application pour garantir une cohérence visuelle.
    \item \textbf{Chargement des données} : Les données utilisées (cours boursiers et indicateurs mondiaux) sont chargées et mises en cache pour accélérer l'application.
    \item \textbf{Fonctions utilitaires} : Fonctions pour formater les dates, créer des badges, normaliser les plages de dates, etc.
    \item \textbf{Composants réutilisables} : Fonctions qui génèrent des blocs d'interface (cartes, sélecteurs, onglets) pour éviter la répétition de code.
    \item \textbf{Callbacks} : Fonctions qui réagissent aux actions de l'utilisateur (sélection de dates, changement d'onglet, choix de métrique) et mettent à jour dynamiquement les graphiques affichés.
    \item \textbf{Lancement de l'application} : Création de l'objet principal Dash et démarrage du serveur web.
\end{itemize}

\subsection*{Fonctionnement de l'application}
L'application propose deux onglets principaux :
\begin{itemize}
    \item \textbf{Séries temporelles} : Permet de sélectionner une plage de dates et d'afficher l'évolution des cours boursiers de deux entreprises (Google et Apple) avant, pendant et après cette plage. Les graphiques sont mis à jour automatiquement selon la sélection.
    \item \textbf{Indicateurs mondiaux} : Permet de choisir un indicateur (population, PIB par habitant, espérance de vie) et d'afficher son évolution par continent sous forme d'histogramme.
\end{itemize}

\subsection*{Points clés pour un néophyte}

\subsection*{Callbacks}
Les callbacks sont des fonctions qui relient l'interface utilisateur aux données et aux graphiques. Ils permettent de rendre l'application interactive :
\begin{itemize}
    \item \texttt{update\_stocks(start, end)} : Met à jour les trois graphiques de séries temporelles et leurs badges en fonction de la plage de dates sélectionnée par l'utilisateur.
    \item \texttt{update\_gap(metric)} : Met à jour le graphique des indicateurs mondiaux selon la métrique choisie (population, PIB, espérance de vie).
    \item \texttt{render\_tab(tab)} : Affiche dynamiquement le contenu de l'onglet sélectionné (séries temporelles ou indicateurs mondiaux) en générant les bons composants.
\end{itemize}
\begin{itemize}
    \item L'application est \textbf{interactive} : chaque action de l'utilisateur (sélection de dates, d'indicateur, changement d'onglet) déclenche une mise à jour immédiate de l'affichage.
    \item L'interface est composée de "cartes" (blocs visuels) pour organiser les éléments de façon claire et agréable.
    \item Les styles (couleurs, dégradés, arrondis) sont centralisés pour garantir une apparence homogène et moderne.
    \item Les données sont chargées une seule fois au démarrage pour de meilleures performances.
    \item Le code est structuré pour être facilement modifiable et réutilisable, grâce à l'utilisation de fonctions génériques pour les composants visuels.
\end{itemize}

\subsection*{Résumé}
Ce code est un bon exemple d'application Dash moderne, claire et efficace, qui sépare bien la logique de présentation, la gestion des données et l'interactivité. Il peut servir de base pour des projets plus complexes ou pour l'apprentissage du développement d'interfaces web interactives en Python.
